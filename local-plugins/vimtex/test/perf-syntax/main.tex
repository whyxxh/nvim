\documentclass{article}
\begin{document}

\noindent Standard \LaTeX{} practice is to write inline math by enclosing it between \verb|\(...\)|:

\begin{quote}
  In physics, the mass-energy equivalence is stated
  by the equation \(E=mc^2\), discovered in 1905 by Albert Einstein.
\end{quote}

\noindent Instead if writing (enclosing) inline math between \verb|\(...\)| you can use \texttt{\$...\$} to achieve the same result:

\begin{quote}
  In physics, the mass-energy equivalence is stated
  by the equation $E=mc^2$, discovered in 1905 by Albert Einstein.
\end{quote}

\noindent Or, you can use \verb|\begin{math}...\end{math}|:

\begin{quote}
  In physics, the mass-energy equivalence is stated
  by the equation \begin{math}E=mc^2\end{math}, discovered in 1905 by Albert Einstein.
\end{quote}

This is a simple math expression \(\sqrt{x^2+1}\) inside text. 
And this is also the same: 
\begin{math}
\sqrt{x^2+1}
\end{math}
but by using another command.

This is a simple math expression without numbering
\[\sqrt{x^2+1}\] 
separated from text.

This is also the same:
\begin{displaymath}
\sqrt{x^2+1}
\end{displaymath}

\ldots and this:
\begin{equation*}
\sqrt{x^2+1}
\end{equation*}

\end{document}
